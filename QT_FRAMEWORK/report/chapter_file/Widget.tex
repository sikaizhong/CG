%----------------------------------------------------------------------------------------
%	CHAPTER 2
%----------------------------------------------------------------------------------------

\chapterimage{title2.jpg} % Chapter heading image

\chapter{Widget}

\section{Qt Quick}\index{Qt Quick}

Qt Quick include:
\begin{enumerate}
	\item QML-Markup language for user interfaces
	\item JavaScript - The dynamic scripting language
	\item Qt C++ - The highly portable enhanced c++ library
\end{enumerate}


\begin{figure}[h]
	\centering\includegraphics[scale=0.4]{hierarchy}
	\caption{Hierarchy}
\end{figure}
In a typical project the front-end is developed in QML/JavaScript and the back-end code, which interfaces with the system and does the heavy lifting\footnote{hard or difficult work} is developed using Qt C++. 

\section{A quick example}\index{A quick example}

\begin{enumerate}
	\item select qt quick application
	\item import QtQuick, which is the common module that will provide basic components (Rectangle, Image, Text) 
	\item import the QtQuick.Window module,which will provide the main window application (Window).
	\item import QtQuick.Layouts if  RowLayout is desired;
	 
\end{enumerate}
Here is an example:

\begin{figure}[h]
	\centering\includegraphics[scale=0.4]{aSimpleExample.jpg}
	\caption{A simple Example}
\end{figure}


\section{QML Syntax}\index{QML Syntax}

\begin{itemize}
	\item  Child elements inherit the coordinate system from the parent,so a X,Y coordinate is always relative to the parent
	\item Every QML file needs to have exactly one root element
	\item Arbitrary elements inside a QML document can be accessed by using their id
	\item Elements can be nested, means a parent element can have child elements. The parent element can be
	accessed using the {\textbf{parent}} keyword
\end{itemize}



\section{Basic Elements}\index{Basic Elements}

\subsection{Item Element}
\subsection{Rectangle Element}
\subsection{Image Element}
\subsection{MouseArea Element}











%------------------------------------------------

\section{Citation}\index{Citation}

This statement requires citation \cite{book_key}; this one is more specific \cite[122]{article_key}.


\section{HyperLink}\index{HyperLink}

This statement requires website hyperlink \href{www.google.com}{google website}; 
this one is local hyperlink \href[page=5]{./reference/1.pdf}{Mastering Qt 5.pdf}.

%------------------------------------------------

\section{Lists}\index{Lists}

Lists are useful to present information in a concise and/or ordered way\footnote{Footnote example...}.

\subsection{Numbered List}\index{Lists!Numbered List}

\begin{enumerate}
\item The first item \cite[122]{ds}
\item The second item
\item The third item
\end{enumerate}

\subsection{Bullet Points}\index{Lists!Bullet Points}

\begin{itemize}
\item The first item
\item The second item
\item The third item
\end{itemize}

\subsection{Descriptions and Definitions}\index{Lists!Descriptions and Definitions}

\begin{description}
\item[Name] Description
\item[Word] Definition
\item[Comment] Elaboration
\end{description}